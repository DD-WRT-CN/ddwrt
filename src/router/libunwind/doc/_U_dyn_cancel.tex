\documentclass{article}
\usepackage[fancyhdr,pdf]{latex2man}

############################################################################
# Shorewall 1.4 -- /etc/shorewall/common.def
#
# This file defines the rules that are applied before a policy of
# DROP or REJECT is applied. In addition to the rules defined in this file,
# the firewall will also define a DROP rule for each subnet broadcast
# address defined in /etc/shorewall/interfaces (including "detect").
#
# Do not modify this file -- if you wish to change these rules, create
# /etc/shorewall/common to replace it. It is suggested that you include
# the command ". /etc/shorewall/common.def" in your
# /etc/shorewall/common file so that you will continue to get the
# advantage of new releases of this file.
#
run_iptables -A common -p icmp -j icmpdef
############################################################################
# NETBIOS chatter
#
run_iptables -A common -p udp --dport 135	  -j DROP
run_iptables -A common -p udp --dport 137:139     -j DROP
run_iptables -A common -p udp --dport 445         -j DROP
run_iptables -A common -p tcp --dport 139         -j DROP
run_iptables -A common -p tcp --dport 445         -j DROP
run_iptables -A common -p tcp --dport 135	  -j DROP
############################################################################
# UPnP
#
run_iptables -A common -p udp --dport 1900	  -j DROP
############################################################################
# BROADCASTS
#
run_iptables -A common -d 255.255.255.255 -j DROP
run_iptables -A common -d 224.0.0.0/4     -j DROP
############################################################################
# AUTH -- Silently reject it so that connections don't get delayed.
#
run_iptables -A common -p tcp --dport 113 -j reject
############################################################################
# DNS -- Silenty drop late replies
#
run_iptables -A common -p udp --sport 53 -mstate --state NEW -j DROP
############################################################################
# ICMP -- Silently drop null-address ICMPs
#
run_iptables -A common -p icmp -s 0.0.0.0 -j DROP
run_iptables -A common -p icmp -d 0.0.0.0 -j DROP





\begin{document}

\begin{Name}{3}{\_U\_dyn\_cancel}{David Mosberger-Tang}{Programming Library}{\_U\_dyn\_cancel}\_U\_dyn\_cancel -- cancel unwind-info for dynamically generated code
\end{Name}

\section{Synopsis}

\File{\#include $<$libunwind.h$>$}\\

\Type{void} \Func{\_U\_dyn\_cancel}(\Type{unw\_dyn\_info\_t~*}\Var{di});\\

\section{Description}

The \Func{\_U\_dyn\_cancel}() routine cancels the registration of the
unwind-info for a dynamically generated procedure.  Argument \Var{di}
is the pointer to the \Type{unw\_dyn\_info\_t} structure that
describes the procedure's unwind-info.

The \Func{\_U\_dyn\_cancel}() routine is guaranteed to execute in
constant time (in the absence of contention from concurrent calls to
\Func{\_U\_dyn\_register}() or \Func{\_U\_dyn\_cancel}()).


\section{Thread and Signal Safety}

\Func{\_U\_dyn\_cancel}() is thread-safe but \emph{not} safe to use
from a signal handler.

\section{See Also}

\SeeAlso{libunwind-dynamic(3)}, \SeeAlso{\_U\_dyn\_register(3)}

\section{Author}

\noindent
David Mosberger-Tang\\
Email: \Email{dmosberger@gmail.com}\\
WWW: \URL{http://www.nongnu.org/libunwind/}.
\LatexManEnd

\end{document}
