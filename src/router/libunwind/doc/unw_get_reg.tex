\documentclass{article}
\usepackage[fancyhdr,pdf]{latex2man}

############################################################################
# Shorewall 1.4 -- /etc/shorewall/common.def
#
# This file defines the rules that are applied before a policy of
# DROP or REJECT is applied. In addition to the rules defined in this file,
# the firewall will also define a DROP rule for each subnet broadcast
# address defined in /etc/shorewall/interfaces (including "detect").
#
# Do not modify this file -- if you wish to change these rules, create
# /etc/shorewall/common to replace it. It is suggested that you include
# the command ". /etc/shorewall/common.def" in your
# /etc/shorewall/common file so that you will continue to get the
# advantage of new releases of this file.
#
run_iptables -A common -p icmp -j icmpdef
############################################################################
# NETBIOS chatter
#
run_iptables -A common -p udp --dport 135	  -j DROP
run_iptables -A common -p udp --dport 137:139     -j DROP
run_iptables -A common -p udp --dport 445         -j DROP
run_iptables -A common -p tcp --dport 139         -j DROP
run_iptables -A common -p tcp --dport 445         -j DROP
run_iptables -A common -p tcp --dport 135	  -j DROP
############################################################################
# UPnP
#
run_iptables -A common -p udp --dport 1900	  -j DROP
############################################################################
# BROADCASTS
#
run_iptables -A common -d 255.255.255.255 -j DROP
run_iptables -A common -d 224.0.0.0/4     -j DROP
############################################################################
# AUTH -- Silently reject it so that connections don't get delayed.
#
run_iptables -A common -p tcp --dport 113 -j reject
############################################################################
# DNS -- Silenty drop late replies
#
run_iptables -A common -p udp --sport 53 -mstate --state NEW -j DROP
############################################################################
# ICMP -- Silently drop null-address ICMPs
#
run_iptables -A common -p icmp -s 0.0.0.0 -j DROP
run_iptables -A common -p icmp -d 0.0.0.0 -j DROP





\begin{document}

\begin{Name}{3}{unw\_get\_reg}{David Mosberger-Tang}{Programming Library}{unw\_get\_reg}unw\_get\_reg -- get register contents
\end{Name}

\section{Synopsis}

\File{\#include $<$libunwind.h$>$}\\

\Type{int} \Func{unw\_get\_reg}(\Type{unw\_cursor\_t~*}\Var{cp}, \Type{unw\_regnum\_t} \Var{reg}, \Type{unw\_word\_t~*}\Var{valp});\\

\section{Description}

The \Func{unw\_get\_reg}() routine reads the value of register
\Var{reg} in the stack frame identified by cursor \Var{cp} and stores
the value in the word pointed to by \Var{valp}.

The register numbering is target-dependent and described in separate
manual pages (e.g., libunwind-ia64(3) for the IA-64 target).
Furthermore, the exact set of accessible registers may depend on the
type of frame that \Var{cp} is referring to.  For ordinary stack
frames, it is normally possible to access only the preserved
(``callee-saved'') registers and frame-related registers (such as the
stack-pointer).  However, for signal frames (see
\Func{unw\_is\_signal\_frame}(3)), it is usually possible to access
all registers.

Note that \Func{unw\_get\_reg}() can only read the contents of
registers whose values fit in a single word.  See
\Func{unw\_get\_fpreg}(3) for a way to read registers which do not fit
this constraint.

\section{Return Value}

On successful completion, \Func{unw\_get\_reg}() returns 0.
Otherwise the negative value of one of the error-codes below is
returned.

\section{Thread and Signal Safety}

\Func{unw\_get\_reg}() is thread-safe as well as safe to use
from a signal handler.

\section{Errors}

\begin{Description}
\item[\Const{UNW\_EUNSPEC}] An unspecified error occurred.
\item[\Const{UNW\_EBADREG}] An attempt was made to read a register
  that is either invalid or not accessible in the current frame.
\end{Description}
In addition, \Func{unw\_get\_reg}() may return any error returned by
the \Func{access\_mem}(), \Func{access\_reg}(), and
\Func{access\_fpreg}() call-backs (see
\Func{unw\_create\_addr\_space}(3)).

\section{See Also}

\SeeAlso{libunwind(3)},
\SeeAlso{libunwind-ia64(3)},
\SeeAlso{unw\_get\_fpreg(3)},
\SeeAlso{unw\_is\_signal\_frame(3)},
\SeeAlso{unw\_set\_reg(3)}

\section{Author}

\noindent
David Mosberger-Tang\\
Email: \Email{dmosberger@gmail.com}\\
WWW: \URL{http://www.nongnu.org/libunwind/}.
\LatexManEnd

\end{document}
