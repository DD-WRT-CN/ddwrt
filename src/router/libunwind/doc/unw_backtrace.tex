\documentclass{article}
\usepackage[fancyhdr,pdf]{latex2man}

############################################################################
# Shorewall 1.4 -- /etc/shorewall/common.def
#
# This file defines the rules that are applied before a policy of
# DROP or REJECT is applied. In addition to the rules defined in this file,
# the firewall will also define a DROP rule for each subnet broadcast
# address defined in /etc/shorewall/interfaces (including "detect").
#
# Do not modify this file -- if you wish to change these rules, create
# /etc/shorewall/common to replace it. It is suggested that you include
# the command ". /etc/shorewall/common.def" in your
# /etc/shorewall/common file so that you will continue to get the
# advantage of new releases of this file.
#
run_iptables -A common -p icmp -j icmpdef
############################################################################
# NETBIOS chatter
#
run_iptables -A common -p udp --dport 135	  -j DROP
run_iptables -A common -p udp --dport 137:139     -j DROP
run_iptables -A common -p udp --dport 445         -j DROP
run_iptables -A common -p tcp --dport 139         -j DROP
run_iptables -A common -p tcp --dport 445         -j DROP
run_iptables -A common -p tcp --dport 135	  -j DROP
############################################################################
# UPnP
#
run_iptables -A common -p udp --dport 1900	  -j DROP
############################################################################
# BROADCASTS
#
run_iptables -A common -d 255.255.255.255 -j DROP
run_iptables -A common -d 224.0.0.0/4     -j DROP
############################################################################
# AUTH -- Silently reject it so that connections don't get delayed.
#
run_iptables -A common -p tcp --dport 113 -j reject
############################################################################
# DNS -- Silenty drop late replies
#
run_iptables -A common -p udp --sport 53 -mstate --state NEW -j DROP
############################################################################
# ICMP -- Silently drop null-address ICMPs
#
run_iptables -A common -p icmp -s 0.0.0.0 -j DROP
run_iptables -A common -p icmp -d 0.0.0.0 -j DROP





\begin{document}

\begin{Name}{3}{unw\_backtrace}{David Mosberger-Tang}{Programming Library}{unw\_backtrace}unw\_backtrace -- return backtrace for the calling program
\end{Name}

\section{Synopsis}

\File{\#include $<$libunwind.h$>$}\\

\Type{int} \Func{unw\_backtrace}(\Type{void~**}\Var{buffer}, \Type{int}~\Var{size});\\

\File{\#include $<$execinfo.h$>$}\\

\Type{int} \Func{backtrace}(\Type{void~**}\Var{buffer}, \Type{int}~\Var{size});\\

\section{Description}

\Func{unw\_backtrace}() is a convenient routine for obtaining the backtrace for
the calling program. The routine fills up to \Var{size} addresses in the array
pointed by \Var{buffer}. The routine is only available for local unwinding.

Note that many (but not all) systems provide practically identical function
called \Func{backtrace}(). The prototype for this function is usually obtained
by including the \File{$<$execinfo.h$>$} header file -- a prototype for
\Func{backtrace}() is not provided by \Prog{libunwind}. \Prog{libunwind} weakly
aliases \Func{backtrace}() to \Func{unw\_backtrace}(), so when a program
calling \Func{backtrace}() is linked against \Prog{libunwind}, it may end up
calling \Func{unw\_backtrace}().

\section{Return Value}

The routine returns the number of addresses stored in the array pointed by
\Var{buffer}. The return value may be zero to indicate that no addresses were
stored.

\section{See Also}

\SeeAlso{libunwind(3)},
\SeeAlso{unw\_step(3)}

\section{Author}

\noindent
David Mosberger-Tang\\
Email: \Email{dmosberger@gmail.com}\\
WWW: \URL{http://www.nongnu.org/libunwind/}.
\LatexManEnd

\end{document}
