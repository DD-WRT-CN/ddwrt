\documentclass{article}
\usepackage[fancyhdr,pdf]{latex2man}

############################################################################
# Shorewall 1.4 -- /etc/shorewall/common.def
#
# This file defines the rules that are applied before a policy of
# DROP or REJECT is applied. In addition to the rules defined in this file,
# the firewall will also define a DROP rule for each subnet broadcast
# address defined in /etc/shorewall/interfaces (including "detect").
#
# Do not modify this file -- if you wish to change these rules, create
# /etc/shorewall/common to replace it. It is suggested that you include
# the command ". /etc/shorewall/common.def" in your
# /etc/shorewall/common file so that you will continue to get the
# advantage of new releases of this file.
#
run_iptables -A common -p icmp -j icmpdef
############################################################################
# NETBIOS chatter
#
run_iptables -A common -p udp --dport 135	  -j DROP
run_iptables -A common -p udp --dport 137:139     -j DROP
run_iptables -A common -p udp --dport 445         -j DROP
run_iptables -A common -p tcp --dport 139         -j DROP
run_iptables -A common -p tcp --dport 445         -j DROP
run_iptables -A common -p tcp --dport 135	  -j DROP
############################################################################
# UPnP
#
run_iptables -A common -p udp --dport 1900	  -j DROP
############################################################################
# BROADCASTS
#
run_iptables -A common -d 255.255.255.255 -j DROP
run_iptables -A common -d 224.0.0.0/4     -j DROP
############################################################################
# AUTH -- Silently reject it so that connections don't get delayed.
#
run_iptables -A common -p tcp --dport 113 -j reject
############################################################################
# DNS -- Silenty drop late replies
#
run_iptables -A common -p udp --sport 53 -mstate --state NEW -j DROP
############################################################################
# ICMP -- Silently drop null-address ICMPs
#
run_iptables -A common -p icmp -s 0.0.0.0 -j DROP
run_iptables -A common -p icmp -d 0.0.0.0 -j DROP





\begin{document}

\begin{Name}{3}{unw\_regname}{David Mosberger-Tang}{Programming Library}{unw\_regname}unw\_regname -- get register name
\end{Name}

\section{Synopsis}

\File{\#include $<$libunwind.h$>$}\\

\Type{const char~*}\Func{unw\_regname}(\Type{unw\_regnum\_t} \Var{regnum});\\

\section{Description}

The \Func{unw\_regname}() routine returns a printable name for
register \Var{regnum}.  If \Var{regnum} is an invalid or otherwise
unrecognized register number, a string consisting of three question
marks is returned.  The returned string is statically allocated and
therefore guaranteed to remain valid until the application terminates.

\section{Return Value}

The \Func{unw\_regname}() routine cannot fail and always returns a
valid (non-\Const{NULL}) string.

\section{Thread and Signal Safety}

The \Func{unw\_regname}() routine is thread-safe as well as safe to
use from a signal handler.

\section{See Also}

\SeeAlso{libunwind(3)}

\section{Author}

\noindent
David Mosberger-Tang\\
Email: \Email{dmosberger@gmail.com}\\
WWW: \URL{http://www.nongnu.org/libunwind/}.
\LatexManEnd

\end{document}
