\documentclass[11pt,a4paper]{article}

\usepackage{graphicx}
\usepackage{url}
%\usepackage{ngerman}
\usepackage[latin1]{inputenc}
\usepackage{textcomp}

\begin{document}

\title{Optimization of Array Bounds Checking in the Context of Cacao Java Virtual Machine}

\author{Br�utigam Klaus, 0825965}

\maketitle

%% Abstract -------------------------------------------------------------------

\abstract{
This document deals with array bounds check optimizations in a very general and historical
way but also strongly relates to the details of implementation within the 
Cacao Java Virtual Machine that was developed by the Institute of Computer Languages
of the Technical University of Vienna. Good and sustainable problem solutions 
can only be found if previous ideas are studied to improve own considerations
and even more important if failures that happened in the past are not repeated.
So the first main part is dedicated to a study of previous
approaches concerning array bounds check optimization and their environments 
to obtain a solid knowledge base for the general problem. 
The second part describes details and relations important
to understand the technique choosen for the Cacao Java VM. So the document should
also be interesting for people only searching for a comprehensive overview of 
the abstract array bounds check optimization problem.

}

%% Introduction ---------------------------------------------------------------

\section{Introduction}
Array bounds checking and the optimizations necessary for efficient
realization were exhaustively researched and discussed in the past four
or five decades. The rise of high level computer programming languages led
directly to several improvements of the background and environments of them.
For example array accesses have always been common malfunction sources and
security holes. If an array access index is not within the bounds of the array,
memory space is accessed illegally. This can lead to dramatic problems.
Therefore computer language architects recognized very fast that
these array accesses must be validated before execution to increase the safety and
stability of applications. So in most cases array access indices are compared 
to the array bounds before executing them. But this introduces some
kind of overhead during compilation and or execution. Due to the fact that
processed information amount increases much faster than our technological
calculation possibilities any type of overhead must be reduced or completely
removed from applications in any thinkable way. As will be seen in the
following chapters the approaches and techniques for reducing overhead of
array bounds checks can be very different. To completely understand the
problem of array bounds checks optimization (short ABCs) all related aspects 
must be taken into account and all topics must also be seen from a more
abstract view.

%% General aspects of ABCs ----------------------------------------------------

\section{A Generic Reflection of ABCs\label{sec:structure}}
This chapter provides a solid overview for ABC optimizations. The
most common forms of implementation are explained and the neccesity for
improvements of ABC optimizations are motivated. Furthermore some background
information will be provided to get a feeling for the relations of the topic.
But it must be kept in mind that an exhaustive explanation of the topic is
far out of the documents scope so we will only touch on the general ideas.

	\subsection{Implementation of Array Bounds Checking}
	Before designing new optimization techniques for ABCs it's important to know
	exactly how these checks are implemented. There are many ways to realize
	ABCs. Elaborateness and performance are important to keep in mind when
	choosing a technique and also the insertion method is interesting because
	it is greatly connected to the language architecture. Should tests
	be integrated automatically without developers knowledge or should
	ABCs only be integrated selectively by programmers or quality assurance.
	
		\subsubsection{Compare Instruction Strategy}
		Most implementations are done using compare instructions of the
		processor and a conditional jump to react to failed comparisons. For this
		technique the array inception and length are obtained and compared to
		the index. For a more detailed description and good examples please
		consult
		\cite{AComparisonOfArrayBoundsCheckingOnSuperscalarAndVLIWArchitectures:2002}.
		For some languages like Java that assure that arrays only start
		with index null only one check is necessary to check lower and 
		upper bounds using an unsigned compare instruction. This assures that
		indices are always within the specified array bounds and that exceptions
		or other reactions are done on the right point of program execution.

		\subsubsection{Hardware Implementation}
		It is also possible to implement ABCs directly and completely in
		hardware. This means some more effort in processor architecture but
		the resulting run-time overhead of ABC execution can be reduced 
		practically to 0.0\%. There have been several approaches from Sun
		Microsystems, Inc. to design a new generation of processors that are
		capable of this. 
		\cite{ProcessorWithAcceleratedArrayAcessBoundsChecking:2000}
		for example describes a processor that uses two hardware comparators and
		a disjunction to inform the instruction execution unit about the validness
		of the array access instruction.
		\cite{ApparatusAndMethodForArrayBoundsCheckingWithAShadowFile:2001} is an
		even more improved way. In addition it uses a shadow register to further
		boost and integrate ABCs into the general instruction flow of the processor.

		\subsubsection{Other Approaches}
		Other more exotic approaches of ABC implementation also exist.
		An interesting one can be found in 
		\cite{ArrayBoundsCheckEliminationUtilizingAPageProtectionMechanism:2003}
		where the processor page protection mechanism is used to identify ABCs and
		even improve their effectiveness by eliminating unnecessary ones.
	
	\subsection{The Real Overhead Introduced by ABCs}
	Before crying out for costly and time-consuming optimizations for a specific
	mechanism we will analyze the overhead caused by this. Is it
	really worthwhile studying and researching the topic? This is in many cases a difficult
	question because future trends must also be considered to get a clear result.
	Concerning ABCs it depends on the ABC implementation, the language
	design and architecture and on the application itself.
	
		\subsubsection{Overhead Reduction through Hardware}
		For decades information processing machines were mainly designed as single
		processor systems that didn't really consider modern programming techniques
		and patterns. Nearly all higher-level improvements that were reached in
		computer programming were done in an abstract way using software
		(virtual machines, multiprocessing, null pointer checks, ABCs\ldots). In the past
		there were some approaches that tried to completely integrate ABCs in hardware, but
		these were refused very fast. Nevertheless we can
		view something in connection to ABCs by comparing benchmarks of the same
		application on the same virtual machines running on different processor
		generations. For example take overhead benchmark results from
		\cite{AVerifiableControlFlowAwareConstraintAnalyzerForBoundsCheckElimination:2009}
		and compare them to that of
		\cite{SafeMultiphaseBondsCheckEliminationInJava:2010} and it's easy to
		recognize that the same application on the same virtual machine with the same
		configuration produces a different overhead. In this case the reason can be
		found very fast. The first benchmark was calculated on a single processor
		system where the second uses a multicore processor. This new processor
		generation is able to the reduce the overhead of ABCs without changing the
		technique or implementation of ABCs. So the question if ABC optimization pays
		not only depends on the software but also on the hardware it is running on.
		Newer processor generations will be able to furthermore reduce the overhead
		raised by ABCs implemented as processor instructions but it is very unlikely
		that the approaches of implementing ABCs completely in hardware will be
		realized in the near future.
		
		\subsubsection{Overhead in Different Types of Applications}
		To reduce overhead of ABCs the application where the optimization operates on
		must use arrays frequently otherwise the overhead produced is insignificant
		and not really measureable\cite{SafeMultiphaseBondsCheckEliminationInJava:2010}.
		An important question is if these optimizations really pay for many
		applications. For example Java is mainly used due to its simple and clear
		object oriented and abstract nature. Most developers use Java only in
		conjunction with powerful frameworks like Spring and many of them only for
		web-based applications. The resulting ABC overhead in these strong object
		oriented applications is practically not measurable even if many arrays
		are used. The overhead is mainly caused by nested method calls (invokes) which
		are also very difficult to reduce. ABC optimizations only really pay on
		applications with simple object relations that make excessive use of arrays
		and matrix calculations like scientific research and development. But for
		these cases it is maybe better and easier to use extremely optimized data
		structures that are recognized by machines and executed in a special way.
		\cite{EliminationOfJavaArrayBoundsChecksInThePresenceOfIndirection:2002}
		mentions some libraries and also explain some new special data structures
		that follow this approach.
		
		\subsubsection{Necessity for ABC Optimization}
		It's difficult to argue pro/contra to ABC optimization and the approach that
		should be taken to implement it because this strongly depends on the language
		architecture and many other factors discussed here. But in general it should
		be the architect's and implementor's target to evolve the language to the most
		effective, efficient and clear state. Even if hardware is capable of executing
		ABCs without any resulting overhead it is better to eliminate unnecessary code
		to improve low-level code quality cleanness and clearance. Program code at
		any abstraction layer should not contain any superfluous instructions or code
		parts even if they have no impact on execution time or security. Moreover some
		slight processor architectures (embedded systems) maybe will never be capable
		of implementing ABC in hardware!

	\subsection{Important Relations to ABC Optimization}
	This chapter explains some important basic conditions that should
	be noticed before starting to think about ABC optimization techniques but
	some of them are valid for nearly all kinds of optimizations. At first the array
	access index types are important to be studied in detail to identify possible
	limitations raised by them. The next important fact is that optimizations
	only pay if their benefits outperform the generated overhead. It should
	also be kept in mind that the elaborateness	and form of implementation
	dramatically influences the effectiveness of the optimization.
	
		\subsubsection{Array Access Index Types}
		The array access index type must also be taken into consideration for
		several reasons. Most computer programming languages (from now on referred
		to as languages) of even all programming paradigms accept the type integer
		as array index type. The word length of this type varies in most of the cases and
		furthermore signs of integral numerical types can also be treated differently.
		In C/C++ integer types exist as signed or unsigned and the length is not specified
		and depends mainly on the instruction length of the underlying hardware. 
		On the other side Java declares its integer type explicitly signed and
		32bit wide (range from -2.147.483.648 to 2.147.483.647 in 2's complement).
		The Java language furthermore declares that overflowing or underflowing values
		are wrapped around which is not desired for array access index types because
		there are hardly reasonable applications that access arrays using overflowing index
		expressions. \cite{JavaLanguageSpecificationv3.0:2005}
		This source of malfunctions and security leaks must also be tackled precisely
		while designing optimization mechanisms.
		
		\subsubsection{Benefits to Execution Time Ratio}
		The ABC optimization benefits must always outperform the execution
		time necessary to calculate them. It is not acceptable that an optimization is
		implemented that saves execution time for some applications and slows others
		down. There is nothing to gain.
		\cite{AComprehensiveApproachToArrayBoundsCheckEliminationForJava:2002}
		for example shows that activated ABC optimization together with others
		slows down some applications of the used benchmark. This is extremely important
		to consider for optimization techniques that work during Just-In-Time
		compilation and execution where expensive, elaborate and in the first
		instance disturbing processes must be omitted as far as possible. As
		mentioned in the next chapter this is maybe best accomplished by splitting
		optimization techniques into different phases to locate them where they are
		unproblematic to run.

		\subsubsection{Intermediate Representation}
		Which intermediate representation should the optimization operate on? This
		has to be considered too. Most compilers fulfill their
		compilation tasks in several phases and for this reason different intermediate
		representations of the programs form. These depend completely on the compiler
		architecture and the language itself especially on the programming paradigm 
		\cite{ModernCompilerImplementationInC:1998}.
		But not only during the compilation process do new program representations
		arise. For interpreted languages some kind of IR is necessary to transport
		code between compiler and interpreter. For example the Java Byte Code is such
		an IR that is delivered to the Java Virtual Machine and executed. If the
		interpreter wants to apply some optimizations to the code it is forced to
		operate on the intermediate representation delivered to him or to do an
		extremely expensive transformation into an other desired form which slows down
		execution even more. Java Byte Code for example is not very adequate to serve
		as IR because it suffers from many problems concerning size, security\ldots 
		and is not really practical for exhaustive optimizations. Finally it should
		not be forgotten that also high level code (Java, C++) can be a very
		suitable representation for obtaining some optimization constraints!\par
		
		A lot of effort has been spent in using Java Byte Code annotations for
		integrating optimization information into conventional class files that are
		ignored by most compilers and allowing some to boost execution performance.
		But these approaches have also a lot of disadvantages because the annotations
		are abused, the usage is not standardized, further increased size and process
		amount\ldots.
		\cite{AFrameworkForOptimizingJavaUsingAttributes:2000} describes an example
		framework that follows the previously discussed one. Also
		\cite{VerifiableAnnotationsForEmbeddedJavaEnvironments:2005} can be an
		interesting source for the beginning because it focuses especially on
		three important optimizations and on low cost checks. Furthermore
		\cite{VerifiableRangeAnalysisAnnotationsForArrayBoundsCheckElimination:2007}
		and \cite{SafeBoundsCheckAnnotations:2008} deal with annotations specifically
		for ABC optimizations. In the historic review many more approaches will be
		mentioned that are based on this construct.\par
		
		Another way to deal with the problem of IR is to use a completely new approach
		and in the past several forms based on Static Single Assignment (SSA) were
		researched and proven to be the commonly used ones in the future. The form is based on
		the constraint that every variable can have only one point of assignment and
		not more or less. But many more constraints and ideas were integrated into
		different modern SSA forms that were adapted and optimized for several
		reasons and techniques. For getting a general idea of SSA forms consult
		\cite{AVerifiableSSAProgramRepresentationForAgressiveCompilerOptimization:2006}.
		\cite{SafeTSAATypeSafeAndReferentiallySecureMobileCodeRepresentationBasedOnStaticSingleAssignmentForm:2001}
		explains SafeTSA which is a type safe and size optimized SSA form and used in
		\cite{UsingTheSafeTSARepresentationToBoostThePerformanceOfAnExistingJavaVirtualMachine:2002}
		to optimize execution performance of a Java Virtual Machine. Another approach
		is covered by
		\cite{AVerifiableSSAProgramRepresentationForAgressiveCompilerOptimization:2006}
		that focuses on clean and efficient embedding of range checks into other aggressive
		optimization phases.
		\cite{ABCDEliminatingArrayBoundsChecksOnDemand:2000} uses a new representation
		by introducing new constraints for dealing efficiently with program branches
		(if-else, while\ldots) called Extended SSA form (eSSA).
		Another SSA form named HSSA is introduced in conjunction with a new
		optimization algorithm ABCE in
		\cite{ArrayBoundsCheckEliminationForJavaBasedOnSparseRepresentation:2009}.	
		Finally
		\cite{TowardsArrayBoundCheckEliminationInJavaTMVirtualMachineLanguage:1999}
		even introduces a new Java Byte Code representation called JVMLa that is also
		suitable for several optimization tasks. Nearly all of these systems are not
		commonly used and are therefore not standardized but show how future
		intermediate representations could be designed or influenced by different
		ideas.

		\subsubsection{Positioning of Optimizations}
		Being generally aware of all previously discussed circumstances another very
		important aspect must be recognized regarding positioning of optimizations.
		The main differentiation can be seen between compiled and interpreted
		languages.\par
		
		Compiled languages are restricted to optimizations that can be calculated at
		compile time (mainly static optimizations). This imposes really big
		limitations on the possible and suitable techniques but expensive
		computations do not therefore really mean big problems. High compile
		time is widely accepted with special optimization level
	 	parameters (for example -O3 for most C compilers). For development and
		debugging the optimizations are not really necessary so they are omitted and
		for deployment one optimized version is created where expensive and long
		running optimizations are not seen problematic.\par
		
		For interpreted language architectures new aspects must be taken into account.
		On the one hand the possibility for powerful highly dynamic optimizations
		arise that use information only available at run time but on the other hand
		it is unacceptable to compute expensive optimizations because it would slow
		down execution dramatically. So the most successful approach would probably be
		to split up dynamic optimizations to compile- and run time parts.
		Expensive precalculations are delegated to the compiler and fast and effective
		optimizations that use information obtained at compile- and run time are
		calculated by the interpreter (virtual machine). This approach was
		exhaustively researched by most recently published ABC optimization
		techniques. At compile time some constraints and equalities/unequalities are
		inferred from high level code and intermediate representations through
		theorem proving and even speculation and special optimizing virtual machines
		interpret these constraints and annotations to calculate very effective
		optimizations efficiently.
		\cite{SafeMultiphaseBondsCheckEliminationInJava:2010} and all previous
		versions of the same authors
		\cite{AVerifiableControlFlowAwareConstraintAnalyzerForBoundsCheckElimination:2009},
		 \cite{SpeculativeImprovementsToVerfifiableBoundsCheckElimination:2008}\ldots 
		are recommended for understanding the complete idea. Also most previously
		mentioned approaches that use annotations in any form (Soot framework)
		basically follow it.\par
		
		One must also consider that optimizations must fit into the compilation and
		interpretation process. Optimizations should not be an isolated process that
		is quickly added to the compiler/interpreter architecture at the end of the
		design process. Moreover optimizations and related techniques and
		especially used intermediate representations should be taken into
		account very early in language architecture design to get clean, clear,
		effective and efficient compilation and interpretation processes overall 
		\cite{ModernCompilerImplementationInC:1998}.
		Nearly all mechanisms researched in the context of Java and the Java Virtual
		Machines suffer from the drawback that many optimizations are added to the
		constructs too late and are therefore not really suitable for the complete
		systems.
		
	\subsection{Types of ABCs}
	ABCs can always be found before array access instructions. In Java the
	instructions are provided for every primitive type as xaload and xastore where
	x represents the array type (integer, float, signed byte, character\ldots).
	But not only the array access instructions themselves are important for ABCs but
	also the context were they appear. Therefore some placements of
	ABCs are of special significance. These are now described a little bit in
	detail to understand which forms of interest can arise. The first three
	classify if ABCs can be proven statically, dynamic or never as really
	necessary during execution for keeping to the Java conventions. The second
	three deal with the code structure where these ABCs occur. A good explanation
	concerning raw classification of ABCs in general can be found in 
	\cite{AFreshLookAtOptimizingArrayBoundChecking:1990} and more recently by the
	same author in
	\cite{OptimizingArrayBoundChecksUsingFlowAnalysis:1993}.
		
		\subsubsection{Redundant ABCs}
		Some ABCs that are inserted by primitive insertion algorithms are completely
		redundant and can be statically proven unnecessary at all. For example three
		sequential array accesses a[i] and a[i+1] and a[i+2] lead to three inserted
		ABCs but the first two of them are completely redundant because the third
		dominates them. The overall number of these checks is very low and the
		execution time amount occupied by them even lower. So there
		are hardly reasonable optimizations possible through elimination of them but
		they are very easy and cheap to compute and remove so it should be done for
		reasons of  cleaness and completeness. These are only important to be
		removed if they occur within loops.
		
		\subsubsection{Partial Redundant ABCs}
		Some ABCs are related through different program branches that arise because of
		if-else or while statements that lead to ABCs that are partially redundant.
		Sometimes they can be simplified or partially eliminated. These ABCs also arise
		relatively often in programs and therefore optimization through
		elimination or code motion can really pay. For a detailed explanation
		on this topic please consult the good considerations in
		\cite{GlobalOptimizationBySuppressionOfPartialRedundancies:1979}.
		
		\subsubsection{Necessary ABCs}
		There are ABCs in existence that exactly represent what they are invented for.
		They are placed in front of array access instructions that can't be proven to
		be valid so there are ABCs that can not be eliminated but sometimes simplified
		or moved using code motion procedures. A good and simple example for this are
		array access indices read from command line or from an external data source
		like files or a database!
		They occur in dependence on the application from seldom to very often. Whether an ABC
		can be proven superfluous or necessary depends stronly on the power of the
		used optimization technique. Sometimes array accesses that deal with
		indirection or interprocedural array access program structures can't be proven
		unnecessary although they are. This will also be explained in the chapter about global ABCs.
		Advanced techniques that are capable of proving these exotic kinds of
		redundant or partial redundant ABCs can sometimes eliminate or simplify big
		amounts of elaborate and unnecessary ABCs.
		
		\subsubsection{Local ABCs}
		Local ABCs are found in straight program structures without branches, loops
		and other conditional statements. These code parts are executed
		sequentially without possible interruptions and are generally very simple and
		short. Local ABCs are in most cases redundant and can be completely eliminated
		but don't have much impact on program execution time.
		
		\subsubsection{Global ABCs}
		Global ABCs are much more difficult. We must at first distinguish between
		interprocedural and intraprocedural global ABCs. If several method invokes are
		integrated into one array access it is hard to automatically analyze the
		structure and obtain constraints especially in languages like Java with
		dynamic class loading. These are called interprocedural. Only very few
		techniques have the power to prove such elaborate ABCs unnecessary. If
		ABCs are only analyzed in the context of one method without
		additional method invokes they are called intraprocedural. Also array accesses
		where indices are again placed in arrays (indirection) can be very challenging
		to prove redudant. A technique that pays special attention to ABCs in the
		presence of indirection is
		\cite{EliminationOfJavaArrayBoundsChecksInThePresenceOfIndirection:2002} that
		gives a solid insight into the difficulties raised by this topic. Due to the
		frequent occurence in combination with matrix calculations and other costly
		procedures optimizations of global ABCs can reduce execution time of
		programs.
		
		\subsubsection{ABCs in Loops}
		ABCs that are placed in loops deserve special attention because here
		the biggest runtime overheads of all possible ABC types arise. If unnecessary ABCs
		are placed in loops they are often repeated and result in huge
		reductions of execution time. Sometimes Local and Global ABCs are found in
		loops and can be eliminated or simplified and sometimes ABCs can even be moved
		out of the loop using code motion techniques. This allows dramatic optimizations.
		One problem that arises is that some optimizations change the program behavior
		and especially the exception throw conventions of the Java language can
		introduce a big additional effort. All this must be taken into account. ABC loop
		optimization is so important and powerful that some techniques reach good
		improvements only by analyzing some common loop structures/patterns. For
		example
		\cite{ArrayBoundsCheckEliminationInTheContextOfDeoptimization:2009} describes
		the mechanisms implemented in the HotSpot Java Virtual Machine that is based
		on this idea.

%% Historic review of ABCs ----------------------------------------------------

\section{A Historic Review of ABCs and Related Optimization}
The presentation of history concerning ABC optimizations will be done in a
chronological order to underline the whole evolution. To get a quick insight into
the different documents some kind of classification will always be provided that
keeps as far as possible to parameters explained in the past chapter so that it
should be possible to quickly understand the raw idea described.

	\subsection{The Early Beginning}
	The first approaches for dealing with ABC optimization all operate on high level
	code representation (no additional intermediate	representation necessary) and
	use data flow graphs to represent program parts in an abstract manner. The real
	ideas of ABCs and optimizations in this direction can't be explicitly
	attributed to one person.\par
	
	In the author's opinion William H. Harrison was the
	first that directly covered the problem. In May 1977 he describes in 
	\cite{CompilerAnalysisOfTheValueRangesForVariables:1977} the more
	general problem of value range analysis of variables. Array bound checks are a
	subproblem of this because array indices are only numerical variables that must
	keep to value ranges (0..arraylength). He introduced in his paper an algorithm
	called Range Propagation that obtains variable value ranges starting
	with [-Inf:+Inf] by analyzing the related programing structures for each
	variable [0..100] for example. Another algorithm Range Analysis even
	covers the problem of induction with variables and therefore mainly concerns
	loops. The algorithm builds through derivations linear equations that are then
	solved based on a data flow analysis. He also introduced the idea of loop
	counting, general program diagnostic and program termination that can identify
	several mistakes and malfunctions to the programmer. This can be seen as the
	real origin of the idea how to insert and optimize ABCs effectively. His
	algorithms are inserted into the conventional compilation process and he even
	considered overall execution time of compilation in his algorithms.\par 
	
	In 1977 nearly at the same time Norihisa Suzuki and Kiyoshi Ishihata composed
	\cite{ImplementationOfAnArrayBoundChecker:1977} where they explicitly insert
	ABCs in high level code in the form of assert statements in front of
	array accesses to provide comfortable error information. This is done by
	deducing formulas from the program and using a theorem prover to decide if an
	ABC (assertion) is necessary for an access instruction or not. Because only static
	analysis is done no real powerful optimizations can be applied. Especially the
	intraprocedural form limits the general success but most local and even some
	global ABCs are tackled. All things considered this paper is the first really
	modern approach for dealing with theorem proving of ABCs and definitly a good
	starting point for studying the topic in depth.\par
	
	In 1979 E. Morel and C. Renvoise published
	\cite{GlobalOptimizationBySuppressionOfPartialRedundancies:1979} that
	introduced the idea of formal consideration of partial redundancies that are
	also necessary for more elaborate analyzing of Global ABCs. For really
	understanding partial redundant ABCs this paper is a must.\par

	In 1982 Victoria Markstein, John Cocke and Peter Markstein presented a new
	approach for optimizing ABCs in 
	\cite{OptimizationOfRangeChecking:1982}. For the first time the idea is
	documented that code parts with high execution frequencies (loops) deserve
	special attention (analysis effort). They introduced usage of subexpression
	elimination and code motion for optimizing ABCs. The technique operates on the
	code by deriving Strongly Connected Regions SCRs using USE and DEF
	functions. These were also used previously by
	\cite{CompilerAnalysisOfTheValueRangesForVariables:1977} albeit the
	explanations were much more complicated and detailed. The technique is based on
	data flow analysis and meant to be integrated into a multiphase compiler after
	all previous optimizations were accomplished. The same authors even obtained an
	USPatent in 1987 for their ABC optimization technique
	\cite{OptimizationOfRangeChecking:1987}.\par
	
	In 1990 Rajiv Gupta published results of his researches on ABC
	optimization in 
	\cite{AFreshLookAtOptimizingArrayBoundChecking:1990}. He used a
	so called Minimal Control Flow Graph MCFG to implement some new
	revolutionary approaches like elimination of redudant ABCs, combination of
	partial redundant ABCs and propagation of ABCs out of loops (code motion). It
	exactly describes how the MCFG is built from source code and also the
	algorithms to perform elimination, propagation and combination of ABCs. This
	gives very precise guidance for first experiments with ABC optimization
	algorithms and abstract representations.
	
	

	In 1992 Jonathan M. Asuru presented in
	\cite{OptimizationOfArraySubscriptRangeChecks:1992} an approach that started dealing
	more precisely with Global ABCs that definitely allow more important
	optimizations. He claimed that previous techniques where not able to deal with
	elaborating loops (nested loops, nonconstant induction variables\ldots) and he
	was right. He described two new mechanisms that were able to deal with much
	more complicated program constructs than all others before.
	\begin{description}
		\item[Inner-Loop Guard Elimination ILGE] deals generally with conventional
		known local and global ABC types.

		\item[Conservative Expression Substitution CES] deals with nested loops and
		partial order for related range checks. Also important with CES is Local Range
		Check Optimization where the greatest lower bound and least upper bounds are
		used to eliminate range checks that are redundant or partially redundant.
		Furthermore Loop Relative Global Range Check Optimization builds up a system
		of structures to improve nested loops checking and finally Range Check
		Propagation uses range check motion to move ABCs to more efficient program
		places.
	\end{description} \par

	In 1993 Rajiv Gupta again published his new insights into the topic in
	\cite{OptimizingArrayBoundChecksUsingFlowAnalysis:1993}. Many refinements to
	his previous ideas were added and the knowledge about the general problem of ABC
	optimization increased. He assumed that readers are familiar with his last work 
	\cite{AFreshLookAtOptimizingArrayBoundChecking:1990}.
	\begin{description}
		\item[Local Optimization] deals with Identical Checks, Subsumed Checks with
		Identical Bounds and Subsumed Checks with Identical Subscript Expressions

		\item[Global Optimizations] are implemented using at first an algorithm for
		modifying checks to create Redundant Checks and secondly an algorithm for the
		reelimination of Redundant Checks. This important idea of first inserting new
		checks and eliminating them is brilliant and will be used by further
		techniques in the future. It allows the identification and resolution of difficult
		redundancies.

		\item[Propagation of Checks Out of Loops] is done using a refined algorithm
		for propagation.
	\end{description} \par

	\subsection{The Chaotic Middle Ages}
	Until 1995 (ca. the first two decades) only a very small group of visioners
	researched ABCs and related optimization techniques. All these approaches
	operated for performance and technological reasons directly on high level
	program code like J. Welsh for Pascal
	\cite{EconomicRangeChecksInPascal:1978} and some others for C and Fortran.
	All things considered all previous approaches were based on the same general
	idea of building flow graphs and applying data flow analysis. But since 1995
	many scientists started researching ABC optimization and therefore a chaotic
	amount of new techniques and ideas appeared. From 1995 on most approaches are
	based on SSA program intermediate representations and algorithms suitable for
	processing optimizations on it but also some exotic and completely new ideas
	were invented and researched. \par
	
	In 1995 Priyadarshan Kolte and Michael Wolfe published
	\cite{EliminationOfRedundantArraySubscriptRangeChecks:1995} where they explain
	their ABC optimization techniques in the Nascent Fortran compiler. They use an
	SSA intermediate form for induction analysis and furthermore a Check
	Implication Graph CIG for global ABCs. They tried different approaches but also
	mentioned that a combination of them would deliver the best results. Also some
	new and more detailed information about Partial Redundant Elimination PRE is
	provided that directly covers optimization of partial redundant ABCs.
	\begin{description}
		\item[No-Insertion NI] means optimization with check elimination without adding
		any new checks that are simpler.
		\item[Safe-Earliest SE] achieves a little bit more eliminations than LNI and NI
		but not much and is the first of two PRE placement checks.
		\item[Latest-Not-Isolated LNI] is only a second PRE placement check.
		\item[Check Strengthening CS] is already described in
		\cite{OptimizingArrayBoundChecksUsingFlowAnalysis:1993}.
		\item[Preheader-Insertion LI] with only loop invariant checks.
		\item[Preheader-Insertion Loop Limit Substitution LLS] is the clear winner if
		compile time and elimination number are taken into account.
	\end{description} \par
	This paper proves in the end that optimized check insertion techniques are much
	better than only elimination techniques. \par
	
	In 1998 Hongwei Xi and Frank Pfenning dealt in 
	\cite{EliminatingArrayBoundCheckingThroughDependentTypes:1998} with ABC
	optimization in ML. So this time ABC optimization also gets very interesting
	for other programming paradigms like functional programming.\par
	But also S. P. Midkiff and J. E. Moreira and M. Snir published a new work
	concerning this topic. 
	\cite{OptimizingArrayReferenceCheckingInJavaPrograms:1998} is a
	very voluminous paper that directly covers the problem of ABC optimization for
	the Java language. At the beginning a very formal theory about the topic is
	presented and then all optimization techniques are explained in detail and
	always presented with good examples on high level Java source code. This work
	gives really comprehensive insight into the topic in relation to the Java
	language. \par
	
	In 1999 the paper
	\cite{TowardsArrayBoundCheckEliminationInJavaTMVirtualMachineLanguage:1999}
	from Xi and Xia introduced a new idea. They did not agree with the standard
	Java Virtual Machine Language JVML (Java Byte Code) and wanted to cope with
	ABCs through a new system of dependent types to improve safeness of the
	language and allow at the same time successful optimizations. So they
	architectured a new intermediate code representation called JVMLa that should
	be created by the Java Compiler and that should only contain ABCs that can't be
	proven unnecessary at compile-time. This approach is questionable because the
	Java Compiler output (JVML) is standardized and wide spread and a very
	complicated ABC optimization part (at compile time) is assumed to function
	already properly in the Java Compiler. This is very inpractical but a theoretic
	idea that may by taken into account by other approaches or languages in the
	future. \par
	
	In 2000 the previously mentioned USPatent
	\cite{ProcessorWithAcceleratedArrayAcessBoundsChecking:2000} was registered and
	Bodik, Gupta and Sarkar presented a new algorithm for optimized ABC execution in
	\cite{ABCDEliminatingArrayBoundsChecksOnDemand:2000}. This algorithm is an
	exotic one because it uses a transformation process at run time that converts
	from Java Byte Code to a slightly extended SSA form (eSSA) which is normaly a
	very expensive procedure and applies a constraint analyzer and solver on it that
	really achieves performance gain. The used constraint analyzer adds
	constraints to the eSSA instructions and builds up an inequality graph IG
	where only a shortest path finder (constraint solver) is applied to optimize
	ABCs. This is very impressive because the whole approach is very costly but it
	really pays like depicted in the results. Array Bound Check elimination on
	Demand (ABCD) was really able to improve execution time of some applications in
	the used benchmark. \par
	
	In 2001, one year after the last USPatent concerning hardware implementation of
	ABCs,
	\cite{ApparatusAndMethodForArrayBoundsCheckingWithAShadowFile:2001} was
	registered as a new idea. Also Chin, Khoo and Xu published in 
	\cite{DerivingPreconditionsForArrayBoundCheckElimination:2001} some further
	ideas how to derive preconditions for ABCs. This very formal approach and
	documentation gives some more detailed insights into the problem of program
	code analysis for optimizing ABCs. \par
	
	In 2002 Bentley, Watterson and Lowenthal compared execution of ABC execution on
	Superscalar and VLIW hardware architecture in
	\cite{AComparisonOfArrayBoundsCheckingOnSuperscalarAndVLIWArchitectures:2002}
	which gives good explanations concerning implementation of ABCs using processor
	compare instructions. \par

	Qian, Hendren and Verbrugge present three algorithms
	they implemented using the Soot framework previously mentioned in IBMs HPCJ in
	\cite{AComprehensiveApproachToArrayBoundsCheckEliminationForJava:2002}. The
	first algorithm Variable Constraint Analysis (VCA) works on a simplified
	directed weighted graph that reflects the relations and behavior of relevant
	variables (array access indices). Array Field Analysis AFA and Rectangular
	Array Analysis deal with special forms of array structures that are hard to
	analyze but can in some circumstances really allow good optimizations. \par
	
	Gurd and Freeman et al. presented a further paper 
	\cite{EliminationOfJavaArrayBoundsChecksInThePresenceOfIndirection:2002} that
	exactly captures the problem of ABCs in the presence of indirection what
	simply means that array access indices are stored in an other array.
	Optimizations in this scope are not easy and require a new technique.
	The solution is not embedded in the Java Compiler or VM itself
	but implemented directly as Java classes that must be used by programmers.
	Three classes with different approaches are presented but only one of them
	seems really useable.
	\begin{description}
		\item[ImmutableIntMultiarray:] The array is checked for invalid bounds and is
	  	not allowed to be modified during index access operations!
		\item[MutableImmutableStateIntMultiarray:] The array can be in two states, in
		the first (mutable) bounds are checked and in the second (immutable) no checks
		occur.
		\item[ValueBoundedIntMultiarray:] This array forces that all entries (indices)
		are within the defined range and so no exception is possible.
	\end{description}
	The idea and all of the explanations and details about matrix
	calculations and related ABCs are really good but generally the practical
	benefits are rare because Java especially wants to permit that program style of
	classes has impact on run time and performance because this is exactly what
	Java wants to abstract from. \par
	
	In 2002 Gupta, Pratt, Midkiff and Moreira registered the USPatent 
	\cite{MethodForOptimizingArrayBoundsChecksInPrograms:2002} where they describe a
	very general method for optimizing ABCs in many facets. \par

	In 2003 Motohiro Kawahito published in an interesting and very exotic idea for ABC
	\cite{ArrayBoundsCheckEliminationUtilizingAPageProtectionMechanism:2003}
	elimination by exploiting the hardware processors page protection mechanism.
	It is based on the idea to place an array to the end/beginning of a page and
	inspect if an access leaves the valid page. If the page protection throws an
	exception it can be derived that the array bounds were violated. \par
	
	In 2005 Nguyen and Irigoin presented with
	\cite{EfficientAndEffectiveArrayBoundChecking:2005}
	a document that implements in a FORTRAN compiler (PIPS) some
	optimizations of ABCs. The general ideas are well known but the detailed
	and formal descriptions give further insights into the details of the
	whole problem. The first idea is based on Gupta's approach. Insert ABCs and
	eliminate redudant ones. Furthermore insert ABCs that can't be proven
	unnecessary. And the last important thing is an algorithm for array size
	checking and analysis to prove array access vioaltions over different procedure
	calls. So the technique allows interprocedural ABC optimization that is a real
	challenge and rarely implemented anywhere. \par
	
	Another large document 
	\cite{SymbolicBoundsAnalysisOfPointersArrayIndicesAndAccessedMemoryRegions:2005}
	was publisehd by Rugina and Rinard where they describe a new framework that
	deals with general problems like access memory regions but also with bounds
	analysis of pointer array indices. Due to the extensiveness of it any
	explanation about this is widely out of this document's scope. \par

	In 2007 two new directions were started that are explained more precisely in the
	next chapter. \par
	
	In 2008 Popeea, Xu and Chin presented in
	\cite{APracticalAndPreciseInferenceAndSpecializerForArrayBoundChecksElimination:2008}
	theory and experiments that use forward relational analysis to obtain
	postconditions and backward precondition derivation to inference precise
	(path, context sensitive) and practical pre/postconditions for an inference
	algorithm to reduce ABCs. Two phases inference and specialization (inserts
	necessary checks) are used to obtain the right pre/postconditions. Weak and
	strong derivation processes lead to the results. In the analysis phase
	Presburger algorithm is used that has high complexity and therefore the number
	of formulas must be reduced as far as possible. Flexivariants of methods mean
	that due to their usage the conditions can be weaker or stronger, so different
	condition strengths are calculated and stored for maximum of efficiency and
	minimum of memory effort. The technique also deals with array indirections.
	It works on a high level artificial language IMP that allows no loops but
	recursion (fix-point analysis) and also has some other restrictions. The inference
	algorithm must be run at compile time and some cases were really expensive
	(\(>\)60min)! So it is not suitable for run-time checks! The approach takes interprocedural
	behavior into account but without dynamic class loading, because the analysis
	works bottom-up. The work is in general an extremely formal approach that makes
	hard use of formulas, foreign symbols and hard to understand mathematical
	definitions so the contribution is in best case of theoretical nature but not
	really suitable in practice.
	
	\subsection{State-Of-The-Art Methodologies}
	The previous path through history was very chaotic but since 2007 two
	real main directions arise that have the potential to be, in the end, satisfactory for
	solving the problem of ABC optimization. The first approach is implemented
	already in the HotSpot Java Virtual Machine and uses a fast transformation from
	Java Byte Code to an SSA based intermediate representation to apply low cost
	but extremely effective analysis techniques for well known code patterns and
	code structures to improve ABC optimization quality quite good and precise. The
	second approach can be summarized under the name Multiphase Bound Check
	Elimination (MBCE) where Java Compiler and Java Virtual Machine process separated
	parts of the ABC optimization process to get high quality and highly efficient results.
	The benchmarks and measurements are also suitable to reveal the real impact and
	efficiency of the presented and tested techniques. \par
	
	In 2007 W�rthinger, Wimmer and M�ssenb�ck published 
	\cite{ArrayBoundsCheckEliminationForTheJavaHotSpotTMClientCompiler:2007} were
	they present insights into the general functionality of the HotSpot Java VM and
	the ABC optimization technique they use. Their algorithm optimizes some well
	known programming patterns in Java to eliminate redundant checks and to move
	partial redundant checks out of loops but keep in mind deopimization for not
	provable check eliminations. This means that if checks can't be proven valid
	the VM stops JIT compilation usage and goes back (ORS) to interpretation mode
	(deoptimization) that is slower but easily allows keeping to the Java language
	conventions like throwing exceptions at the right place\ldots. But results show
	no significant increase in optimization, because the part of execution time is
	not very high in the used benchmarks! A maximum of 5\% was possible! The
	solution is clean, keeps to widely accepted standards and optimizes ABCs if
	they occur very often. \par
	
	Another publication 
	\cite{VerifiableRangeAnalysisAnnotationsForArrayBoundsCheckElimination:2007}
	from 2007 is the beginning of the most recently researched
	technique of von Ronne, Psarris and Niedzielski. This document describes
	verifiable annotations based on SafeTSA for optimizing ABCs by shifting slow
	optimizations to a Java to SafeTSA compiler and let the JIT compiler do fast
	and also efficient optimizations based on the created annotations. This allows
	usage of expensive optimization techniques in time critical virtual machines
	(JIT compilers). The interesting problem of integer
	overflow/underflow that is not taken into account by many other techniques but
	means a security and safety leak is also mentioned. This should be taken into consideration for
	any implementation to react accordingly to over- and underflows. The theory of
	the build constraints and linear inequality/equalities and the prover are
	strongly related to 
	\cite{ABCDEliminatingArrayBoundsChecksOnDemand:2000} but provide
	some improvements and are also meant to be only integrated into low level code
	(SafeTSA) as annotations and therefore use much more expensive techniques than
	ABCD can. \par
	
	In 2008 Nguyen and Keryell published
	\cite{EfficientIntraproceduralArrayBoundChecking:2008} that describes a
	further framework that is based on deriving expensive linear equalities at
	compile-time and integrating them as annotations into a special form of
	intermediate representation like SafeTSA for example. This approach is strongly
	connected to the previous paper
	\cite{VerifiableRangeAnalysisAnnotationsForArrayBoundsCheckElimination:2007}
	(same introduction). They implemented a verifyable annotation generator that
	is based on Jikes RVM 2.2.0. Its use extended Fourier-Motzkin's variable
	elimination (FMVE) to derive proofs. The framework presented is called VBCE
	Verifiable Bound Check Elimination and can be directly compared to VDF (Verifiable
	Data Flow Analysis). The intraprocedural approach that is not able to deal with
	interprocedural relations, works on high level code at compile time and
	annotations and is therefore not suitable for direct JIT optimization.
	Loop invariants are not taken into account. \par
	
	Furthermore in this year Gampe, von Ronne and Niedzielski et al. published two
	updates
	\cite{SafeBoundsCheckAnnotations:2008}
	and
 	\cite{SpeculativeImprovementsToVerfifiableBoundsCheckElimination:2008}
 	of their previous work
 	\cite{VerifiableRangeAnalysisAnnotationsForArrayBoundsCheckElimination:2007}
 	that introduce speculation and newer mechanisms for annotations into the
 	optimization technique. The papers improves the previous framework further.
 	It also works on high level code at compile but now takes loop invariants into
 	account.
 	
 	To get the relationship of their publications on the
 	topic these papers should be studied at least at a glance.
	
	In 2009 W�rthinger, Wimmer and M�ssenb�ck revealed a further update of their
	approach in 
	\cite{ArrayBoundsCheckEliminationInTheContextOfDeoptimization:2009} that
	introduces further improvements and optimizations in the HotSpot Java VM.
	The document is logically strongly related to
	\cite{ArrayBoundsCheckEliminationForTheJavaHotSpotTMClientCompiler:2007}
	and explains that code duplication (loop versioning) is avoided using the
	interpretation mode of the VM to retain exception semantics for PRCs.
	It provides a good explanation of all parts especially PRC in loops and why loops
	with additional exit conditions can't be optimized. Furthermore conditions are
	not managed in an equality/inequality graph but for every instruction.
	Interprocedural optimizations are not taken into account due to dynamic class
	loading mechanism of Java. Loading new classes leads to more information about
	array accessing and can lead to deoptimization (OSR at safepoint back to stack)
	if the new conditions can't prove the previous optimizations. But several
	ABCs can be grouped if proveable which leads to further optimizations. All
 	things considered a really suitable ABC optimization technique clearly
 	embedded in the VM and very effective going by the test results. \par
 	
 	\cite{ArrayBoundsCheckEliminationForJavaBasedOnSparseRepresentation:2009} from
 	Yang, Huang and Yang is another publication from 2009 that introduces new
 	algorithm ABCE that is based on HSSA form and can be used in a Java static
 	compiler. It builds an inequality graph and eliminates FRC (fully redudant
 	checks) and PRC (partially redundant checks) on it in two phases. The work is
 	based on the Openjc java compiler that is restricted to very few platforms and
 	not widespread. This compiler uses a five phase generation style where in the
 	fourth phase the optimization is done on code that was transformed to HSSA 
 	form and back. That means a lot of time effort. The quality and embedding of
 	the algorithm into the compiler is suitable but not useable for run time
 	optimizations and therefore too restricted. Furthermore the article only uses
 	percentage of ABC elimination as benchmark and a very intransparent
 	performance score graphic! \par
 	
 	In the same year Gampe, von Ronne and Niedzielski et al. published another
 	update
 	\cite{AVerifiableControlFlowAwareConstraintAnalyzerForBoundsCheckElimination:2009}
 	of their previous work
 	\cite{VerifiableRangeAnalysisAnnotationsForArrayBoundsCheckElimination:2007}
 	that further improves their ABC optimization technique. In the document a new
	more modern technique is presented that is also based on eSSA of the ABCD
	algorithm and claimed to be 10\% more efficient than on SafeTSA. Then CAS
	(constraint analysis system) a verifiable symbolic program constraint analyser
	is presented. The approach is like previous ones also based on multiphase and
	FVME. \par
	
	From 2010 the last known publication about ABC optimization is
	\cite{SafeMultiphaseBondsCheckEliminationInJava:2010} again from  Gampe, von
	Ronne and Niedzielski et al. They refined the details of their
	multiphase ABC optimization technique based on SafeTSA with annotations, CAS
	and elaborate speculations. The results and also the additional provided
	information about the topic show that the solution is practically satisfying.
	The most challenging problems
	connected to ABCs are solved and the solution is well embedded into a Java
	Compiler and Java VM. They are not widespread but solve the problem of ABCs as
	far as possible to be both efficient and convenient.

	\subsection{Historical Conclusion}
	Lastest research in ABC optimizations prove the assumption that only an early and
	clean integration into the compilation and or interpretation process together
	with suitable intermediate representations and multiple optimization phases 
	lead to practically satisfying results that really pay and tackle all newly
	introduced problems. For Java the lately improved approach of the HotSpot Java 
	VM and the multiphase bound check elimination MBCE promise practical success
	because all other previous techniques suffer from many drawbacks that can't
	really be solved due to the overall environment of the language architecture,
	hardware, future developments and other aspects of reality.

%% ABCs in the Cacao Java VM --------------------------------------------------

\section{ABCs in Cacao Java Virtual Machine}

	\subsection{Introduction to Cacao Java Virtual Machine}
	The work on the Cacao Java VM was started in 1996 by Reinhard Grafl and
	Andreas Krall. They wanted to create a free, efficient and stable Java VM for
	a wide variety of processor architectures and platforms. In the following years
	many people joined the development team and implemented several extensions,
	refinements and more and more architectures were provided. The implementation
	is mostly done in C and some parts were already refactored to C++. The VM is partitioned
	into modules (*.h/*.hpp with *.c/*.cpp) and keeps to the relevant Java standards 
	concerning behavior and general structure. For more detailed information about the
	Cacao Java VM consult
	\cite{HandbookForJavaCacaoVirtualMachine:1996} and
	\cite{CacaoA64BitJavaVMJustInTimeCompiler:1997}.

	\subsection{Array Bounds Checking in Cacao Java Virtual Machine}
	In this chapter we will describe in general all necessary aspects concerning ABCs.
	The Cacao Java VM uses integer compare instructions of the processor to implement
	ABCs. They are generated together with all other code parts during JIT compilation
	by the code generators from a special internal intermediate representation whose form
	is strongly related to quadruples and derived from the Java Byte Code provided by 
	Java class files. ABCs are positioned in front of array accesses. 
	The JIT compilation is done locally on every method invocation and therefore only
	allows intraprocedural optimizations, so it's impossible to solve intermediate ABC
	optimization because all underlying and parallel methods must be analyzed. Some
	experiments with example applications showed that execution with ABCs turned on leads
	to significant execution time overhead. For some applications this overhead is
	enormous so in consideration of all properties of the Cacao Java VM, its environment and
	implementation an ABC optimization would be able to drastically reduce execution
	time for some types of applications. In the past this motivated the creation of
	an ABC optimizer in the Cacao Java VM.
	
	\subsection{Array Bounds Check Optimization in Cacao Java Virtual Machine}
	In the year 1998 Christopher Kr�gel added an ABC optimizer for loops to the Cacao Java VM. This led to
	improvements concerning execution time of array intensive applications. For the
	reasons previously explained the optimization algorithm design was limited by 
	many constraints so the author only concentrated on the most important topics.
	A useful generic introduction into the optimizer can be obtained from
	\cite{HandbookForJavaCacaoVirtualMachine:1996} in chapter 5.7 Loop Optimization.
	
		\subsubsection{Power and Properties of the ABC Optimizer}
		The intermediate representation (IR) used within the Cacao Java VM is not really suitable
		for complicated high performance optimization like for example SSA based IRs are. This
		has historical reasons, because these representation formats got famous in the past decade
		and also have disadvantages so that their usage must be carefully taken into account
		during the whole design process of an abstract machine or compiler.
		For this reason the algorithm design was in the first instance limited by the complexity raised 
		by analysis and transformations of this quadruple related IR. The time consumption of the 
		optimizer must be recognized.
		Secondly the JIT compiler architecture limited the optimization scope to one local method.
		This further constrains possibilities because no global ABCs can be removed.
		Due to the internal architecture of the Cacao Java VM
		the structures and algorithms grew very complicated and so the developer only focused on the
		most important facts concerning ABC optimization. The author concentrated only on optimization within 
		loops and even furthermore he restricted his algorithm to loops with induction 
		variables with constant grow factors (up or down). Maybe
		the algorithm designer used the argument that many empirical tests in the past showed that these
		kinds of loops introduce most of the execution time overhead because they are very often and commonly
		used by programmers in many programming languages. The results he presented proved the concept to
		be generally suitable. So he did not analyze detailed theoretical differences on redudant/%
		partial redudant or necessary/local/global ABCs. He simply concentrated on ABCs that seem to 
		cause the most overhead in loops within normal applications.

		\subsubsection{Generic Design Issues of the ABC Optimizer}
		This chapter will provide a generic overview over the design of the ABC optimizer 
		and the added improvements. A detailed explanation is, due to the amount of information,
		not possible within the document's scope. For a further functional explanation please consult
		\cite{HandbookForJavaCacaoVirtualMachine:1996}
		and for the implementational details directly the comments in the Cacao Java VM source code. 
		\par
		The general idea of the algorithm is to analyze a method during its JIT compilation process
		for loops. This is accomplished by a depth first search of code blocks (basic blocks), then
		calculating the dominators of the blocks using the Lengauer-Tarjan algorithm 
		\cite{AFastAlgorithmForFindingDominatorsInAFlowGraph:1979}
		and detecting all loops by searching for back edges
		\cite{ModernCompilerImplementationInC:1998}. These are very time consuming tasks because all
		nodes (basic blocks) of the method must be iterated several times one every method compilation.
		After this happened identical loops are merged, nested loops resolved and all together they are topologically sorted.
		The loops are optimized in their sorted order by analyzing their loop headers and the variables/constants
		involved in array access instructions. If properties are found that allow optimization the
		loop and its header are adapted to suit optimal execution behavior. If nested loops exist their
		combinatorical behavior must be considered.
		The loop header must be adapted in such a way that new dynamic and static constraints introduced by code motion
		out of the loop are inserted. In general only loops can be optimized where array index variables
		are constant (full), only increment (lower bound) or only decrement (upper bound) by constant values.
		Loop headers with or statements can't be optimized by the algorithm as cases
		where array index variables are modified within exception handling (catch) blocks.
		The original loops are replaced by the optimized (memory layout) with a conditional
		jump to the originals if some dynamic constraints don't hold.
		Also important for the optimization design is to notice that the original version of the loop is
		held in memory parallel to the optimized one to allow correct execution with ABCs in the case
		that some dynamic constraints don't hold or can't be proved. This introduces a lot of time
		and space overhead during jit compilation because of the copy procedure. The correct patching
		of links between code blocks and an exact exception handling add even more complexity to the
		solution. So every optimized loop exists two times within the memory.

		\subsubsection{Implementation of the ABC Optimizer}
		
		This chapter is dedicated to give an abstract but fundamental introduction
		to the structure of the ABC optimization algorithm. At first
		the most important data structures are explained to get an idea what kind
		of data is necessary for the processes (temporary and in the end). Then we will
		have a clear look at the algorithm in form of an abstract pseudocode that represents
		generally all important functions and the behavior overall. The functions are
		named to be self explanatory to express their meaning but for understanding the
		ideas behind it maybe a good beginning would be to first consult the good generic 
		textual description found in
		\cite{HandbookForJavaCacaoVirtualMachine:1996} in chapter 5.7 Loop Optimization.
		This could give some useful introducing information before studying the real behavior.

		The most important data structures to understand the ABC optimization algorithm:
		
		\begin{description}

			\item[LoopData (ld)] is a data structure that contains all necessary
				intermediate information and states necessary during the ABC
				optimization like detected loops, exception graphs, auxiliary
				information to improve algorithm performance...

			\item[JITData (jd)] contains all information about the just-in-time
				compiled method that is currently optimized. This structure
				contains for example the methods code blocks (basic blocks),
				the number of codeblocks, exception table...

			\item[LoopContainer (lc)] that holds all information about a specific
				loop, like the contained code blocks for example.

			\item[BasicBlocks (bb)] is a container that contains code of one
				atomic code part of the method (no branch/jump) instruction
				contained. So this means that a basic block is guaranteed to
				be executed in one exact sequence.

		\end{description}
		
		So the most important data structures are now identified and the algorithm
		can be described in pseudocode.\\

		Preconditions are that the method must be completely parsed, analyzed
		and all information must be correctly stored within the related
		jitdata structure.

		%% Codesection -----------------------------------------------------------------

		\begin{verbatim}

			depthFirstSearchOnMethodBasicBlocks (jd)
			exploreDominators (ld)
			detectLoopsInMethod (jd, ld)
			analyzeDoubleLoopHeaders (ld)
			analyzeNestedLoops (jd, ld)
			for all LoopContainers (lc) {
			    optimizeSingleLoop (jd, ld, lc)
			}
			redirectMethodEntryBlock (jd, ld)

			optimizeSingleLoop (jd, ld, lc) {
			    analyzeLoopForArrayAccesses (jd, ld, lc)
			    if ( analyzeLoopForOrAndExceptions (jd, ld, lc) == true ) {
			        initializeLoopHeaderConstraints (jd, ld, lc)
			        removeHeaderBoundChecks (jd, ld, lc)
			        for all BasicBlocks (bb) {
			            removeBoundChecks (jd, ld, lc)
			        }
			        insertChecksIntoLoopHeader (jd, ld)
			    }
			}

			initializeLoopHeaderConstraints (jd, ld, lc) {
			    if ( last Instruction in header is comparison against constant )
			        obtainLeftArgumentDetails (ld, lc)
			    } else if ( last Instruction in header is standard comparison ) {
			        obtainLeftArgumentDetails (ld, lc)
			        obtainRightArgumentDetails (ld, lc)
			    } else {
			        // unknown constraints for these conditions
			    }
			    if ( both sides contain changing or both have no index variables ) {
			        // no constraints can be proven valid
			    } else {
			        // force index variable to be on left side
			        ForceRightSideToBeConstant (ld, lc)
			        obtainDynamicConstraints (ld, lc)
			    }
			}

			removeBoundChecks (jd, ld, lc) {
			    mergeConstraintsWithNestedLoops (ld, lc)
			    for all Instructions of BasicBlock (bb) {
			        if ( Instruction is accessing array ) {
			            obtainIndexVariableDetails (ld, bb)
			            calculateBoundConstraints (jd, ld, bb)
			            createLoopHeaderCodeMotionChecks (jd, ld, bb)
			        }
			    }
			    for all successors of BasicBlock (bb) {
			        removeBoundChecks (jd, ld, bb)
			    }
			}

			insertChecksIntoLoopHeader (jd, ld) {
			    copyCompleteLoop (jd, ld)
			    createBasicBlockForCodeMotionChecks (jd, ld)
			    insertGotoInstructionsForHandlingNewLoopHeaderBasicBlock (jd, ld)
			    adaptExceptionHandlingRegions (jd, ld)
			    informParentLoopsAboutNewHeader (jd, ld)
			    for all LoopHeaderCodeMotionChecks {
			        createInstructionsForConstraintChecks (jd, ld)
			        insertInstructionsIntoNewLoopHeader (jd, ld)
			    }
			    updateJumpTargetsOfLoops (jd, ld)
			    updateExceptionHandlingForNewAndDuplicatedLoop (jd, ld)
			}
		
		\end{verbatim}
		
		%% Codesection end -------------------------------------------------------------

		Postconditions are that the method's jitdata structure holds correct, executable
		and optimized loop code blocks that have no semantic differences to the original form.\newline
		
		Keep in mind that this pseudocode representation is due to the documents scope
		very generic so for more detailed information it's maybe best to directly consult
		source code with its comments because it's hard to go further into details and
		preserving overview.
		
		\subsubsection{Results and Performance of the ABC Optimizer}
		This chapter presents results that were taken from an application that sieves prime numbers.

		\begin{table}[!ht]
			\caption{Benchmark Results of ABCs in Cacao}
			\label{BenchResABCsInCacao}
			\begin{center}
			\begin{tabular}{lccc}
				Configuration&Light&Normal&Hard\\
				\hline
				Runs&10&5&2\\
				Normal&5,563&55,195&551,355\\
				without ABCs -cb&5,562&55,192&551,414\\
				with -oloop&5,560&55,182&551,350\\
				\hline
			\end{tabular}
			\par\medskip\footnotesize
			The measurements are selected to rise by a factor of 10 every level starting from 50000/10000.
			\end{center}
		\end{table}
		
		From this data with polynomic behavior a fixed percentage of speed acceleration should
		be proven but it's easy to see that there are no differences between these configurations.
		This encourages the assumption that ABCs are always created within the Cacao Java VM whatever
		arguments are passed. The proof for this assumption will be done using assembler code output
		from the machine application.
		
		\subsubsection{Possible Improvements for the ABC Optimizer}
		The algorithm is due to its constraints of the virtual machine architecture a very conservative approach
		and has no really complicated mechanisms like speculation/probability calculations or exhaustive
		stack exploration. The implementation is done in C (structured programming technique) so
		it's not really likely to find exhaustive performance improvements in its implementation, but like
		always some possibilities remain that could lead to better results. A very expensive
		part of the ABC optimization algorithm includes the necessity for loop detection at the beginning. 
		This is accomplished by usage of the Lengauer-Tarjan algorithm presented in
		\cite{AFastAlgorithmForFindingDominatorsInAFlowGraph:1979}. 
		This algorithm has good properties concerning asymptotic complexity and is especially 
		used by mathematicians therefore the preferred solution for very big graphs (\(>\) 30.000 nodes)
		and also presented for example in 
		\cite{ModernCompilerImplementationInC:1998}. 
		But in practice methods and nested loops are in general much smaller and own only
		tens or hundreds of nodes in their control flow graphs. Cooper, Harvey and Kennedy showed in
		\cite{ASImpleFastDominanceAlgorithm:2001} 
		that an other algorithm that has higher asymptotic complexity than the Lengauer-Tarjan performs much 
		better for smaller graphs because the complexity issue is only really valid for very big (asymptotic) graphs 
		that normally do not occur in most other programs. The authors claim that their algorithm is for
		conventional method and loop graphs 2.5 times faster than the Lengauer-Tarjan algorithm. If the
		experiments are valid, there exists a possibility to improve the whole ABC optimization algorithm by
		using the simpler algorithm of Cooper, Harvey and Kennedy. The algorithm and especially its proof 
		are also much simpler to understand and therefore to implement. \par
		One fact must be kept in mind. To really allow drastic improvements in algorithm performance 
		the complete Cacao Java Virtual Machine architecture must be reconsidered and this demanding
		task can only be done in an iterative way over a long period of time.

%% Conclusion -----------------------------------------------------------------

\section{Conclusion}
The document presented general information about the topic of ABCs overall and also preserved
historical aspects that show up the evolvement. The main part dealed with ABC optimization in the
Cacao Java VM. All things considered the task of successful array bounds check optimization is a complicated and
challenging quest with many facets and relations. A lot of details must be considered in design and
also implementation phases. But it is definitly a problem that must be solved satisfactorily for all
future developments in computer languages and informatics in general because it is simply not bearable
to waste computing time for any superfluous calculations for any reasons. The ABC optimization within loops in
the Cacao Java Virtual Machine is very conservative due to its environment and also for historical reasons.
So we are definitly far away from the perfect solution but the algorithm proved to be useful in some cases
(programs) and this is the first step towards evolvement of a practical really satisfying result.

%% Bibliography ---------------------------------------------------------------

\bibliographystyle{alpha}
\bibliography{bibfile}

\end{document}

%% End ------------------------------------------------------------------------
